\documentclass[12pt]{amsart}
\usepackage{jeffmath}
\usepackage[letterpaper]{geometry}
\geometry{verbose,tmargin=2.5cm,bmargin=2.5cm,lmargin=2.5cm,rmargin=2.5cm}
\usepackage[unicode=true, bookmarks=true,bookmarksnumbered=false,bookmarksopen=false, breaklinks=false,pdfborder={0 0 1},backref=false,colorlinks=false]{hyperref}
\hypersetup{pdfauthor={Jeffrey Schenker}}
\numberwithin{equation}{section}
\numberwithin{figure}{section}
\theoremstyle{theorem}
\newtheorem{theorem}{\theoremname}
%biblatex
%\usepackage[backend=biber, sorting=nyt,sortcites=true,style=trad-abbrv,date=comp,abbreviate=true,hyperref=true, url=false,arxiv=abs,doi=false,isbn=false]{biblatex}
%\addbibresource{/users/jeffrey/Library/texmf/bibtex/library.bib}
%\DeclareFieldFormat{titlecase}{#1}
%\bibliography{/users/jeffrey/Library/texmf/bibtex/library.bib}
%\AtEveryBibitem{\clearfield{month}}
%\DeclareSourcemap{
%  \maps[datatype=bibtex, overwrite]{
%    \map{
%      \step[fieldset=month, null]
%    }
%  }
%}



\begin{document}
	\title{Quantum channels on qbits}
	\maketitle
	\section{A little linear algebra}
	Consider the vector space $\cH = \bbC^n$.  An 
	element $\vec{z}\in \cH$ will be denoted in bold, where 
	$\vec{z}=(z_1,\ldots,z_n)$.  We will use the Dirac notation wherein
	$$\ket{\vec{z}} \ = \ \begin{bmatrix} z_1 \\ \vdots \\ z_n \end{bmatrix}$$
	denotes the column vector and  $$\bra{\vec{z}} \ = \ \begin{bmatrix} z_1^* 
		& \cdots & z_n^*  \end{bmatrix} $$
	denotes the complex conjugate row vector.  The inner product between two 
	vectors is given by:
	$$\bra{\vec{z}}\ket{\vec{w}} \ = \ \sum_{j=1}^n z_j^* w_j \ . $$
	Recall the complex conjugate $(a+ib)^* = a - i b$, for a complex number.  
	Note that
	$$\bra{\vec{z}}\ket{\vec{w}}^* \ = \ \bra{\vec{w}}\ket{\vec{z}} \ . $$
	The \emph{norm} $\norm{\vec{z}}$ of a vector, is it's Euclidean length, 
	given by
	$$\norm{\vec{z}}^2 \ = \ \sum_{j=1}^n |z_j|^2 \ = \ 
	\diracip{\vec{z}}{\vec{z}} \ . $$
	
		Let $\vec{A}$ be an $n\times n$ matrix,  
	$$ \vec{A} \ = \ \begin{bmatrix}
		A_{11} & \ldots & \ldots & A_{1n} \\
		\vdots & \ddots & & \vdots \\
		\vdots & & \ddots & \vdots \\
		A_{n1} & \ldots & \ldots & A_{nn}
	\end{bmatrix} \ . $$
	The \emph{transpose} $\vec{A}^T$ of $\vec{A}$ is the matrix with 
	$\vec{A}^T_{ij}\ = \ A_{ji}$.  The complex conjugate of the transpose is 
	called the \emph{adjoint} of $\vec{A}$ and denoted $\vec{A}^\dagger$.  
	Thus, 
	$$A_{ij}^\dagger \ = \ A_{ji}^T \ . $$
	Note that 
	$$\left ( \dirac{\vec{z}}{\vec{A}}{\vec{w}} \right )^* \ = \ 
	\dirac{\vec{w}}{\vec{A}^\dagger}{\vec{z}} \ . $$
	A matrix $\vec{A}$ is \emph{self-adjoint} if  $\vec{A}^\dagger = \vec{A}$, 
	i.e., $A_{ji} = A_{ij}^*$.  For a self-adjoint matrix, one has
	\begin{equation}\label{eq:selfadjointconjugateip}
		\left ( \dirac{\vec{z}}{\vec{A}}{\vec{w}} \right )^* \ = \ 
		\dirac{\vec{w}}{\vec{A}}{\vec{z}} 
	\end{equation}
	
	Recall that an \emph{eigenvector} of a matrix $A$ is a non-zero vector 
	$\vec{z}$ such that
	$$ A \ket{\vec{z}} \ = \ \lambda \vec{z}$$
	with $\lambda \in \bbC$, called the \emph{eigenvalue} of $\vec{z}$.  
	Suppose that $\vec{A}$ is self-adjoint.  Then
	$$\dirac{\vec{z}}{\vec{A}}{\vec{z}} \ = \ \lambda 
	\diracip{\vec{z}}{\vec{z}} \ = 
	\ \lambda \norm{\vec{z}}^2 \ . $$
	Taking complex conjugates, and using \eqref{eq:selfadjointconjugateip}, we 
	find that
	$$ \lambda^* \norm{\vec{z}}^2 \ = \ \dirac{\vec{z}}{\vec{A}}{\vec{z}} \ = \ 
	\lambda \norm{\vec{z}}^2 \ . $$
	Since $\norm{\vec{z}}\neq 0$, we conclude that $\lambda^*=\lambda$, i.e. 
	that $\lambda$ is a real number.  So every eigenvector of a self-adjoint 
	matrix is real.
	
	A sequence of $m$-vectors $\vec{z}_1,\ldots,\vec{z}_m$ is called 
	\emph{ortho-normal} if 
	$$\diracip{\vec{z_i}}{\vec{z_j}} \ = \ \delta_{i,j} \ . $$ 
	Here $\delta_{i,j}$ is the \emph{Kronecker $\delta$ symbol}; it equals $1$ 
	if $i=j$ and $0$ if $i\neq j$.  One can show that any ortho-normal sequence 
	is linearly independent.  If $m=n$, then the sequence is also a basis, and 
	is called an ortho-normal basis.  If $\vec{z}_1,\ldots,\vec{z}_n$ is an 
	ortho-normal basis and $\vec{z}\in \cH$, then
	$$\ket{\vec{z}} \ = \ \sum_{j=1}^n \left ( \diracip{\vec{z_j}}{\vec{z}} 
	\right ) \ket{\vec{z_j}} \ .  $$
	Another way to write this expression is to note that the \emph{identity 
	matrix} $\vec{I}$, with $I_{ij}=\delta_{ij}$, can be written
	$$ \vec{I} \ = \ \sum_{j=1}^n \ket{\vec{z_j}}\bra{\vec{z_j}} \ . $$
	
	\begin{thm} Every self-adjoint matrix $\vec{A}$ has an ortho-normal basis 
	of eigenvectors.
		\end{thm}
	\begin{proof} First note that every matrix $\vec{A}$ has at least one 
	eigenvector.  This is because $\lambda\in \bbC$ is an eigenvalue of 
	$\vec{A}$ if and only if $\det (\lambda \vec{I} - \vec{A}) \ = \ 0 $.  The 
	determinant $\det (\lambda \vec{I} - \vec{A})$ is a polynomial of degree 
	$n$ in $\lambda$. The Fundamental Theorem of Algebra states that every 
	polynomial has a root in $\bbC$.  For a self-adjoint matrix, the eigenvalue 
	has to be real, of course.
		
		Let $\vec{z}_1$ be an eigenvector of $\vec{A}$, with eigenvalue 
		$\lambda_1$.  Divide $\vec{z}_1$ by its length if necessary so that it 
		has length one. Consider the set of vectors orthogonal to $\vec{z}_1$:
		$$\cH_1 \ = \ \{ \vec{z} \ : \ \diracip{\vec{z_1}}{\vec{z}} = 0 \} \ .$$
		For such vectors,
		$$\dirac{\vec{z_1}}{\vec{A}}{\vec{z}} \ = \ \lambda_1  
		\diracip{\vec{z_1}}{\vec{z}} \ = \ 0 \ . $$
		That is $\vec{A}$ maps the space  $\cH_1$ into itself.  Choosing an 
		ortho-normal basis for $\cH_1$ we can rexpress $\vec{A}$ on this space 
		as an $n-1 \times n-1$ matrix (because $\dim \cH_1=n-1$).  One can show 
		that the matrix is self-adjoint.  So it has an eigenvector $\vec{z}_2$ 
		and eigenvalue $\vec{\lambda}_2$.  We can take $\vec{z}_2$ to have 
		length one. Now define $\cH_2$ to be all the set 
		of all vectors orthogonal to $\vec{z}_1$ and $\vec{z}_2$.  If $n=2$ the 
		only such vector is $\vec{0}$, but but if not there is a non-trivial 
		subspace.  Repeating the above argument we can find an eigenvector 
		among them.  Continue this process until you have $n$ eigenvectors, at 
		which  point we have a basis which is ortho-normal by construction.
		\end{proof}
	
		If $\vec{A}$ is self-adjoint, with ortho-normal eigenvectors 
		$\vec{z}_1,\ldots,\vec{z}_n$ and eigenvalues 
		$\lambda_1,\ldots,\lambda_n$, then 
		\begin{equation}\label{eq:Aevevrep}
			\vec{A} \ = \ \sum_{j=1}^n \lambda_j \ket{\vec{z}_j} 
			\bra{\vec{z}_j} \ . 
			\end{equation} 
		For any complex valued function $f$ on the eigenvalues, one defines
		\begin{equation}\label{eq:f(A)}
			f(\vec{A}) \ = \ \sum_{j=1}^n f(\lambda_j) \ket{\vec{z}_j} 
			\bra{\vec{z}_j} \ .
		\end{equation}
	\section{Finite dimensional quantum systems} 
	We will 
	consider \emph{finite dimensional quantum systems}.  These are 
	approximations to real systems, which typically need infinite dimensional 
	spaces for their full description.  However, in many interesting systems 
	there is a finite number of the dimensions needed to describe the 
	``effective,'' or important, properties of the system.
	
	For each system we consider we will take the Hilbert space $\cH=\bbC^n$ for 
	some $n$. An \emph{observable} of our quantum system is an $n\times n$ 
	self-adjoint matrix.   Recall that a matrix $\rho$ is \emph{positive 
	semi-definite} if
	\begin{equation}\label{eq:psd}
		\dirac{\vec{z}}{\rho}{\vec{z}} \ \ge \ 0 
	\end{equation}
for all $\vec{z}\in \cH$.  Such a matrix is self-adjoint with all non-negative 
eigenvalues.  Conversely, if $\rho$ is self-adjoint and all of its eigenvalues 
are non-negative, then $\rho$ is positive semi-definite. 
	A \emph{state} of our system is a positive semi-definite matrix $\rho$ with 
	$\tr \rho =1$.  Here $\tr \vec{A}$ denotes the \emph{trace} of a matrix, 
	$\tr\vec{A} \ = \ \sum_{i=1}^n A_{ii}$.  One can also has
	$$\tr \vec{A} \ = \ \sum_{j=1}^n \dirac{\vec{z}_j}{\vec{A}}{\vec{z}_j}$$
	for any ortho-normal basis $\vec{z}_1,\ldots,\vec{z}_n$.
	
	According to quantum mechanics, the eigenvalues 
	$\lambda_1,\ldots,\lambda_n$ of an observable $\vec{A}$ represent the 
	possible outcomes of a measurement of $\vec{A}$. If our system has state 
	$\rho$ and we measure $\vec{A}$, quantum mechanics predicts that we obtain 
	the answer $\lambda_j$ with probability 
	$\dirac{\vec{z}_j}{\rho}{\vec{z}_j}$, where $\vec{z}_j$ is the 
	corresponding eigenvector. Since $\sum_j \dirac{\vec{z}_j}{\rho}{\vec{z}_j} 
	= \tr \rho = 1$, the probabilities add to one.  This all means that if we 
	prepare our system in state $\rho$ many times and measure $\vec{A}$ each 
	time, we will obtain different answers, but in the long run the fraction of 
	times we get $\lambda_j$ as an answer will approach 
	$\dirac{\vec{z}_j}{\rho}{\vec{z}_j}$. 
		
	
	\section{The Bloch Sphere representation of qubits}
	A \emph{qubit} is a ``two-level'' quantum sytem.  The Hilbert space of a 
	qbit is $\bbC^2$, which is the simplest non-trivial Hilbert space. An 
	observable for a qubit is a $2\times 2$ self-adjoint matrix.  Any such 
	matrix $\vec{A}$ is of the form
\begin{equation}\label{eq:sa22}
	\vec{A} \ = \ \begin{bmatrix}
		\alpha & \gamma - i \delta  \\
		\gamma + i \delta & \beta 
 	\end{bmatrix} 
\end{equation}
 with $\alpha,\beta,\gamma,\delta \in \bbR$.  The space of observables 
 is a four dimensional vector space with a basis given by the identity
 $$\sigma_0 = I = \begin{bmatrix} 1 & 0 \\
 	0 & 1 \end{bmatrix} $$
 together with the \emph{Pauli} matrices:
 \begin{equation}\label{eq:Pauli}
 	\sigma_1 = \begin{bmatrix} 0 & 1 \\
 		1 & 0 \end{bmatrix} \ , \quad \sigma_2 = \begin{bmatrix} 0 & -i \\
 		i & 0 \end{bmatrix}\ , \quad \text{and} \quad \sigma_3 = 
 	\begin{bmatrix} 1 
 		& 0 \\ 0 & -1 
 	\end{bmatrix} \ .
 \end{equation}
Explicitly, if $\vec{A}$ is given by \eqref{eq:sa22}, then 
$$\vec{A} \ = \ \frac{\alpha + \beta}{2} I + \gamma \sigma_1 + 
\delta \sigma_2  + \frac{\alpha -\beta}{2} \sigma_3 \ . $$

A density matrix of the qubit is a $2\times 2$ positive semi-definite 
	matrix.  Such a matrix  is  the form \eqref{eq:sa22} with 
	with $\alpha,\beta\ge 0$, $\alpha+\beta=1$, and $\det \rho = 
	\alpha \beta -\gamma^2-\delta^2 \ge 0$.  A density matrix $\rho$ 
	represents a \emph{pure state} if $\rho$ is a rank one projection.  For 
	qubits, 
	this happens precisely if the determinant $ab-\gamma^2-\delta^2 =0$.  
	Then we have 
	$\gamma+i \delta =\e^{i\theta }\sqrt{ab}$ for 
	some $\theta \in [0,2\pi)$, and so
	$$\rho \ = \ \begin{bmatrix} a & e^{-i\theta} \sqrt{ab} \\
		\e^{i\theta} \sqrt{ab} & b \end{bmatrix} \ = \ 
	\begin{bmatrix}\sqrt{a} 
		\\
		\e^{i\theta} \sqrt{b} \end{bmatrix} \begin{bmatrix} \sqrt{a} & 
		\e^{-i\theta} \sqrt{b} \end{bmatrix} \ . $$
	\begin{prop}[Bloch sphere representation] Let $\rho$ be the density matrix 
		of a state of $\mathfrak{A}_2$.  Then $\rho$ has the form
		\begin{equation}\label{eq:Blochsphere}
			\rho \ = \ \tfrac{1}{2} \left [ I + \vec{v}\cdot \vec{\sigma} 
			\right ] \ = \ \tfrac{1}{2}\left [ I + v_1 \sigma_1 + v_2 \sigma_2 
			+ v_3 
			\sigma_3 \right ]  \ ,
		\end{equation}
		with $\vec{v}=(v_1,v_2,v_3)$ and $|\vec{v}|^2 \le 1$.  Furthermore, 
		$\rho$ 
		is a pure state if and only 
		if $|\vec{v}|=1$. 
	\end{prop}
	\begin{proof}
		Let $\rho$ have the form \eqref{eq:sa22} with $\alpha,\beta\ge 0$, 
		$\alpha+\beta=1$ and $\alpha \beta - \gamma^2 -\delta^2\ge 0$.  Taking 
		$v_1= 2\gamma$, $v_2=2 \delta $ and $v_3 = \alpha-\beta$, we find that
		$$\rho \ = \ \tfrac{1}{2}\left [ I + v_1 \sigma_1 + v_2 \sigma_2 + v_3 
		\sigma_3 \right ]  \ , $$
		with 
		$$v_1^2 + v_2^2 + v_3^2 \ = \ 4 (\gamma^2 + \delta^2) + \alpha^2 - 2 
		\alpha\beta + \beta^2 \ \le \ (\alpha+\beta)^2 \ 
		= \ 1 \ . $$
		It also follows that $\rho$ is pure, with $\alpha \beta 
		-\gamma^2-\delta^2=0$, if and only if 
		$|\vec{v}|^2=1$. 
	\end{proof}

\section{The Pauli matrices} The Pauli matrices introduced above have a very 
nice algebraic relation with one another.  First of all, one sees that
$$ \sigma_j^2 \ = \ I \quad \text{for } j=1,2,3 \ . $$
Secondly, one has 
\begin{multline*} \sigma_1 \sigma_2 \ = \ i \sigma_3 \ , \quad \sigma_2 
\sigma_3 \ = \ i \sigma_1 \ , \quad \sigma_3 \sigma_1 \ = \ i \sigma_2 \ , \\
	\sigma_3 \sigma_2 \ = \ - i 
	\sigma_1 \ , \quad \text{and} \quad \sigma_2 \sigma_1 \ = \ -i \sigma_3 \ , 
	\quad  \sigma_1 \sigma_3 \ = \ -i \sigma_2 \  .
	\end{multline*}
These identities can be summarized as
\begin{equation}\label{eq:pauliproducts}
	\sigma_{i}\sigma_j \ = \ \delta_{ij} I + i \sum_{k=1}^3 \eps_{ijk}\sigma_k
\end{equation}
where $\eps_{ijk}$ is the \emph{totally antisymmetric symbol}
\begin{equation}\label{eq:eps}
	\eps_{ijk} \ = \ \begin{cases} 0 & \text{if any two indices are equal,} \\
		1 & \text{if } (i,j,k) = (1,2,3), \ (2,3,1),\ \text{or}\ (3,1,2) \ , 
		\text{and} \\
		-1 & \text{if } (i,j,k) = (3,2,1), \ (2,1,3),\ \text{or}\ (1,3,2) \ .
	\end{cases}
\end{equation}

In terms of the \emph{commutators} $[\sigma_i,\sigma_j] \ = \ \sigma_i \sigma 
_j -\sigma_j \sigma_i$, \eqref{eq:pauliproducts} implies the following 
\emph{commutation relations}
\begin{equation}\label{eq:paulicommutators}
	[\sigma_i,\sigma_j] \ = \ 2i \sum_{k=1}^3 \eps_{ijk} \sigma_k \ .
\end{equation}
That is 
\begin{multline*}
	[\sigma_1,\sigma_2]\ = \ - [\sigma_2,\sigma_1] \ = \ 2i\sigma_3 \ , \quad 
	[\sigma_2,\sigma_3]\ = \ -[\sigma_3, \sigma_2] \ = \ 2i \sigma_1 \ , \\
	\text{and} \quad [\sigma_3,\sigma_1] \ = - [\sigma_1,\sigma_3] \ = \ 2i 
	\sigma_2 \ .
\end{multline*}

\section{Evolution of an isolated Qubit}
To do some interesting physics we need to understand how the state of our 
system evolves in time.   Let us first consider a qubit all by itself.  In that 
case the state $\rho(t)$ of the system as a function of time $t$ satisfies the 
\emph{Liouville-Schr\"odinger} equation
\begin{equation}\label{eq:LiouvilleSchrodinger}
	\frac{d}{dt} \rho(t) \ = \ - i [H,\rho(t)] \ ;
\end{equation}
on the right hand side we have the commutator $[H,\rho(t)] = H\rho(t) -\rho(t) 
H$. Here $H$ is a special observable, called the \emph{Hamiltonian}, which 
corresponds to the energy of the system. The equation should have a factor of 
$\frac{1}{\hbar} = \frac{2\pi}{h}$ where $h$ is Planck's constant; by choosing 
the right units we can take $\hbar =1$.  
	
To solve \eqref{eq:LiouvilleSchrodinger} we can use the commutation relations 
\eqref{eq:paulicommutators}:  

\begin{exer}Suppose that $H = c I + \vec{w}\cdot \vec{\sigma}$, where 
$\vec{w}\in \bbR^3$ 
	and that $\rho \ = \ \frac{1}{2}[I + \vec{v}\cdot \vec{\sigma}] $, where 
	$|\vec{v}| \le 1$. Show that
	$$[H,\rho] \ = \ i (\vec{w}\times \vec{v}) \cdot \vec{\sigma} \ , $$
	where 
	$$ \vec{w} \times \vec{v} \ = \ (w_2v_3 - w_3v_2, \ w_3v_1-w_1v_3, \ 
	w_1v_2-w_2v_1) $$
	is the cross product.
	\end{exer}

Note that the constant term $cI$ in $H$ drops out of the commutator.  This is a 
usual fact from physics: the origin of the energy doesn't matter; it is energy 
\emph{differences} that are physical.  So we can assume that $H=\vec{w}\cdot
\vec{\sigma}$.  If $\rho(t) \ = \ \frac{1}{2}[I + \vec{v}(t)\cdot 
\vec{\sigma}]$, then  the Liouville-Schr\"odinger equation 
\eqref{eq:LiouvilleSchrodinger} becomes
\begin{equation}\label{eq:BlochLS}
	\frac{d}{dt} \vec{v}(t) \ = \ \vec{w} \times \vec{v}(t) \ . 
\end{equation}
Eq.\ \eqref{eq:BlochLS} is a a linear differential equation for 
$\vec{v}(t)$; given an initial value $\vec{v}(0)$ it is well known that there 
is a unique solution to \eqref{eq:BlochLS} for all $t\in \bbR$.  This equation  
describes the rotation of 
$\vec{v}(t)$ around the axis $\hat{\vec{w}} = \frac{1}{|\vec{w}|}\vec{w}$, 
with angular speed $\omega=|\vec{w}|$. 
\begin{exer} Let $\vec{v}_0$ be a point in the Bloch sphere 
$\{|\vec{v}|\le 1\}$ and consider the solution $\vec{v}(t)$ to 
\eqref{eq:BlochLS} with $\vec{v}(0) = \vec{v}_0$. 
 Show that 
$$\vec{v}(t) \ = \ \vec{v}_0 + ( \cos(\omega t) -1)\vec{u} + \sin(\omega t) 
\vec{u}' \ , $$
where 
$ \vec{u} =  \vec{v}_0 - ( \hat{\vec{w}}\cdot \vec{v}) 
\hat{\vec{w}}$
is the projection of $\vec{v}_0$ onto the plane orthogonal to $\vec{w}$,  and 
$\vec{u}' = \hat{\vec{w}}\times \vec{u}$.
\end{exer}
So a qubit by itself has a vector $\vec{v}(t)$ in the Bloch sphere which just 
rotates around a given axis at a certain speed.  

\section{Quantum Channels} Things can get more 
interesting if we allow the qubit to interact with its environment.  In general 
it is quite a hard problem to solve for the evolution of a large quantum 
system, such as a qubit together with its environment.  However, there is a 
useful approximation, called the \emph{Markov approximation}, 
  which roughly speaking applies if the environment is so big, and complicated, 
  that we can neglect the effect of the system on its environment.  You 
  don't really need to worry about what that all means.  In the end it comes 
  down to the following: \emph{the evolution of a system interacting with an 
  environment over a discrete interval of time is given by a \emph{quantum 
  channel}, which is a \emph{completely positive, trace preserving} map on 
  density matrices.} 

We should define the terms that appear in the above assertion.
\begin{defn} A \emph{completely positive map} on the states of a quantum system 
is a map 
	\begin{equation}\label{eq:Kraus}
		\Phi(\rho) \ = \ \sum_{j=1}^m A_j \rho A_j^\dagger
	\end{equation}
	where $A_1,\ldots,A_m$ are certain $n\times n$ matrices, called the 
	\emph{Kraus operators of the map} .
	\end{defn}
In fact, the correct mathematical definition of a completely positive map is 
 more abstract;  the fact that any such map  (on a finite dimensional 
system) has the form \eqref{eq:Kraus} is called \emph{Kraus' Theorem}.  
\begin{defn}  A map $\rho \mapsto T(\rho)$ defined on matrices is \emph{trace 
preserving} if $\tr \rho = \tr T(\rho)$.
	\end{defn}
Putting together the two definitions we have
\begin{defn} A \emph{quantum channel} is a trace preserving, completely 
positive map on the states of a quantum system.
	\end{defn}

Recall that the trace $\tr \rho = \sum_i \rho_{ii}$.  If $A$ and $B$ are two 
matrices, it follows that
\begin{equation}\label{eq:cyclicity}
	\tr A B \ = \sum_{i}\left(\sum_j A_{i,j}B_{j,i}\right) \ = \ \sum_j 
	\left(\sum_i B_{j,i}A_{i,j}\right) \ = \tr B A \ .
\end{equation}
This fact is called \emph{cyclicity of the trace}.  For a quantum channel 
$\Phi$ and a density matrix $\rho$, we have
$$1 = \tr \Phi(\rho) \ = \ \sum_{j=1}^m \tr A_j\rho A_j^\dagger \ = \ 
\sum_{j=1}^m \tr A_j^\dagger A_j \rho \ = \ \tr \left(\sum_{j=1}^m A_j^\dagger 
A_j \right) \rho \ . $$
Since $\rho$ could be any density matrix, this holds if and only if 
$$\sum_{j=1}^m A_j^\dagger A_j \ = \ I .$$
So \emph{a quantum channel is a completely positive map with Kraus operators 
that satisfy $\sum_j A_j^\dagger A_j$.}

\section{Amplitude Damping Channel of a Qubit}
A simple \emph{amplitude damping} of a qubit has the form
\begin{equation}\label{eq:ADChannel}
	\Phi_\gamma(\rho) \ = \ A_0 \rho A_0^\dagger + A_1 \rho A_1^\dagger 
\end{equation}
where 
\begin{equation}\label{eq:simpleAs}
	A_0 \ = \ \begin{bmatrix}
	1 & 0 \\ 0 & \sqrt{1-\gamma}
\end{bmatrix} \quad \text{and} \quad A_1 \ = \ \begin{bmatrix}
0 & \sqrt{\gamma} \\ 0 & 0 
\end{bmatrix} \ . 
\end{equation}
It represents the ``relaxation'' of a qubit to a ground state given by 
$\rho_{GS} \ = \ \left[ \begin{smallmatrix}
	1 & 0 \\ 0 & 0 
\end{smallmatrix}\right] $ .  
\begin{exer}	
For $\rho = \frac{1}{2} [I +\vec{v}\cdot \vec{\sigma}]$, show that 
$\Phi_\gamma(\rho) \ = \ \frac{1}{2} [I +\vec{v}'\cdot \vec{\sigma}]$ with
\begin{multline*}
	\vec{v}' \ = \  \left(\sqrt{1-\gamma} v_1, \sqrt{1-\gamma}v_2 , \gamma + 
(1-\gamma)v_3\right) \\ = \ \sqrt{1-\gamma}(\vec{v} - (\vec{e}_3\cdot \vec{v}) 
\vec{e}_3) + (\gamma+ (1-\gamma) \vec{e}_3\cdot \vec{v}) \vec{e}_3 \ ,
\end{multline*}
where $\vec{e}_3=(0,0,1)$.
\end{exer}


Note that, for $A_0$ and $A_1$ as in \eqref{eq:simpleAs}, we have
$$A_0 \ = \ \frac{1}{2} \left[(1+\sqrt{1-\gamma})I + 
(1-\sqrt{1-\gamma})\sigma_3\right]\quad \text{and} \quad A_1 \ = \ 
\frac{\sqrt{\gamma}}{2} \left[\sigma_1 + i \sigma_2\right] \ . $$
Using this observation, one can define an amplitude damping channel with 
respect to any axis 
$\hat{\vec{w}}$ (in place of $\vec{e}_3$).
\begin{exer}
	Consider the map on states given by $ \frac{1}{2} [I +\vec{v}\cdot 
	\vec{\sigma}] \mapsto [I +\vec{v}'\cdot \vec{\sigma}]$ where
	$$\vec{v}' \ = \ \sqrt{1-\gamma}(\vec{v} - (\hat{\vec{w}}\cdot \vec{v}) 
	\hat{\vec{w}}) + (\gamma+ (1-\gamma) \hat{\vec{w}}\cdot \vec{v}) 
	\hat{\vec{w}} \ . $$
	Show that this map corresponds to the channel $\Phi_{\gamma,\hat{\vec{w}}}$ 
	given by \eqref{eq:ADChannel} with
	$$ A_0 \ = \ \frac{1}{2}\left[(1+\sqrt{1-\gamma})I + 
	(1-\sqrt{1-\gamma})\hat{\vec{w}}\cdot \vec{\sigma}\right] \quad \text{and} 
	\quad A_1 \ = \ \frac{\sqrt{\gamma}}{2} \left[ \vec{w}^\perp_1\cdot 
	\vec{\sigma} + i \vec{w}^\perp_2\cdot \vec{\sigma}  \right] \ ,$$
	where $\hat{\vec{w}}_1^\perp$ and $\hat{\vec{w}}_2^\perp$ are two vectors 
	such that
	\begin{enumerate}
		\item $\hat{\vec{w}}_j^\perp$, $j=1,2$, are orthogonal to 
		$\hat{\vec{w}}$, i.e $\hat{\vec{w}}_j^\perp \cdot \hat{\vec{w}} = 0$, 
		$j=1,2$, and
		\item $\hat{\vec{w}}_1^\perp \times \hat{\vec{w}}_2^\perp = 
		\hat{\vec{w}}$.
	\end{enumerate} 
\end{exer}

Returning to the simple amplitude damping channel with $A_0$, $A_1$ as in 
\eqref{eq:simpleAs}, note that if we write $\vec{v}=(v_1,v_2,v_3)$ as a row 
vector, then the channel amounts to the map
$$\vec{v} \ \mapsto \ \gamma \vec{e}_3 + \vec{v} \begin{bmatrix} 1-\gamma & 0 & 
0 \\ 0 & \sqrt{1-\gamma} & 0 \\
	0 & 0 & \sqrt{1-\gamma} \end{bmatrix} \ . $$
\begin{exer} Show that the result $\Phi_\gamma^n(\rho)$ of $n$ repeated 
applications of $\Phi_\gamma$ to $\rho=\frac{1}{2}[I + \vec{v}\cdot 
\vec{\sigma}]$ is 
given by
$$\vec{v}  \ \mapsto \ \gamma \left(\sum_{j=0}^{n-1} (1-\gamma)^j \right) 
\vec{e}_3 + \vec{v} 
\begin{bmatrix} (1-\gamma)^{n} & 0 & 
	0 \\ 0 &( 1-\gamma)^{\frac{n}{2}} & 0 \\
	0 & 0 & ( 1-\gamma)^{\frac{n}{2}} \end{bmatrix} \ . $$
\end{exer}
Recall the geometric series
$$\sum_{j=0}^{\infty} (1-\gamma)^j \ = \ \frac{1}{\gamma} \ , $$
for $0 < \gamma \le 1$.  In particular,
$$\sum_{j=0}^{n-1} (1-\gamma)^j \ = \ \frac{1}{\gamma} - \sum_{j=n}^\infty 
(1-\gamma)^j \ = 
\ \frac{1}{\gamma} - \frac{(1-\gamma)^n}{\gamma} \ . $$
\begin{exer} For a general amplitude damping channel 
$\Phi_{\gamma,\hat{\vec{w}}}$ show that $\Phi_{\gamma,\hat{\vec{w}}}(\rho)\ = 
\ 
\frac{1}{2}[I + \vec{v}_n\cdot 
	\vec{\sigma}]$ with 
	$$ \vec{v}_n \ = \ \hat{\vec{w}} + (1-\gamma)^n ( \hat{\vec{w}}\cdot 
	\vec{v} -1 )\hat{\vec{w}} + (1-\gamma)^{\frac{n}{2}} (\vec{v}- 
	(\hat{\vec{w}}\cdot \vec{v}) \hat{\vec{w}}) \ . $$
	\end{exer}
Note that $\vec{v}_n = \hat{\vec{w}} + O((1-\gamma)^{\frac{n}{2}})$, so the 
state converges exponentially fast to the pure state $\frac{1}{2}[I + 
\hat{\vec{w}}\cdot \vec{\sigma}]$. 

Finally let us introduce a combination of the free dynamics of the qubit and an 
amplitude damping channel.  Let $\Phi_{\gamma,\hat{\vec{w}},\delta}$ be the 
channel of the form \eqref{eq:ADChannel} with
\begin{multline*}
	A_0 \ = \ \frac{1}{2}\left[(1+\sqrt{1-\gamma})I + 
(1-\sqrt{1-\gamma})\hat{\vec{w}}\cdot \vec{\sigma}\right] \e^{i\delta 
\hat{\vec{w}}\cdot \sigma} \\ \text{and} \quad  A_1 \ = \ 
\frac{\sqrt{\gamma}}{2} \left[ \vec{w}^\perp_1\cdot 
\vec{\sigma} + i \vec{w}^\perp_2\cdot \vec{\sigma}  \right] \e^{i \delta 
\hat{\vec{w}}\cdot \vec{\sigma} } \ ,
\end{multline*}
with $\vec{w}_j^\perp$, $j=1,2$, as above.
This represents the evolution of a qubit subject to amplitude damping and a 
Hamiltonian $- \hat{\vec{w}}\cdot \vec{\sigma}$ over a time interval of length 
$\delta$. The equilibrium state is still $\frac{1}{2}[I + \hat{\vec{w}}\cdot 
\vec{\sigma}]$, but the evolution is different in that the component 
perpendicular to $\hat{\vec{w}}$ rotates as it shrinks to zero. 
\begin{exer}Consider the generalized amplitude damping channel 
$\Phi_{\gamma,\hat{\vec{w}},\delta}$.  Show that for 
$\rho=\frac{1}{2}[I + \vec{v}\cdot \sigma]$, we have
$\Phi_{\gamma,\hat{\vec{w}},\delta}^n(\rho) = \frac{1}{2} [I + \vec{v}_n\cdot 
\sigma]$ with
$$ \vec{v}_n \ = \ (1 +(1-\gamma)^n(\hat{\vec{w}}\cdot \vec{v} - 
1))\hat{\vec{w}} +  (1-\gamma)^{\frac{n}{2}} (\cos(n\delta) \vec{u} + 
\sin(n\delta) \vec{u}') \ , $$
where $\vec{u} \ = \ \vec{v} - (\hat{\vec{w}}\cdot \vec{v})\hat{\vec{w}}$ and 
$\vec{u}'= \hat{\vec{w}}\times \vec{u}$. 
\end{exer}

\section{Ergodic sequences of quantum channels}






\end{document}




